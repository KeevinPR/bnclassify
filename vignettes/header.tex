
\newcommand{\rtbl}[1]{Table~\ref{tbl:#1}}
\newcommand{\req}[1]{Equation~\ref{eq:#1}} 
\newcommand{\rsec}[1]{Section~\ref{sec:#1}}
\newcommand{\rfig}[1]{Figure~\ref{fig:#1}} 
\usepackage{centernot}
\newcommand{\bigCI}{\mathrel{\text{\scalebox{1.07}{$\perp\mkern-10mu\perp$}}}}
\newcommand{\nbigCI}{\centernot{\bigCI}} 
\newcommand{\CI}{\mathrel{\perp\mspace{-10mu}\perp}}
\newcommand{\nCI}{\centernot{\CI}}

\DeclareMathOperator*{\argmax}{arg\,max}

\def\X/{\ensuremath{\mathbf{X}}}
\def\x/{\ensuremath{\mathbf{x}}}
\newcommand{\ith}[1]{\ensuremath{#1^{(i)}}} 
\def\coef/{\boldsymbol{\beta}} 
\def\mthetas/{\ensuremath{\boldsymbol{\theta}}} 
\def\gstuc/{\ensuremath{\mathcal{G}}}
\def\pcgx/{\ensuremath{P(c\mid\mathbf{x})}}
\def\pcxemp/{\ensuremath{\widehat P(c, \mathbf{x})}}
\def\pcx/{\ensuremath{P(c, \mathbf{x})}}
\def\PCX/{\ensuremath{P(C, \mathbf{X})}}
\def\PCGX/{\ensuremath{P(C\mid\mathbf{X})}}
% implementation of rjournal commands
% re-implement code to let is be used with default pandoc template

\newcommand{\kbd}[1]{{\normalfont\texttt{#1}}}
\newcommand{\key}[1]{{\normalfont\texttt{\uppercase{#1}}}}
\DeclareRobustCommand\samp{`\bgroup\@noligs\@sampx}
\def\@sampx#1{{\normalfont\texttt{#1}}\egroup'}
\newcommand{\var}[1]{{\normalfont\textsl{#1}}}
\newcommand{\file}[1]{{`\normalfont\textsf{#1}'}} 
\newcommand{\code}[1]{\texttt{#1}}
\let\option=\samp
\newcommand{\dfn}[1]{{\normalfont\textsl{#1}}}
% \acronym is effectively disabled since not used consistently
\newcommand{\acronym}[1]{#1}
\newcommand{\strong}[1]{\texorpdfstring%
{{\normalfont\fontseries{b}\selectfont #1}}%
{#1}}
\let\pkg=\strong
\newcommand{\CRANpkg}[1]{\href{https://CRAN.R-project.org/package=#1}{\pkg{#1}}}%
\let\cpkg=\CRANpkg
\newcommand{\ctv}[1]{\href{https://CRAN.R-project.org/view=#1}{\emph{#1}}}
\newcommand{\BIOpkg}[1]{\href{https://www.bioconductor.org/packages/release/bioc/html/#1.html}{\pkg{#1}}}

\def\tecvin/{"technical" vignette}
\def\invin/{"introduction" vignette}
