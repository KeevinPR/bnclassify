\section{Background}\label{background}

\label{sec:bcground}

\subsection{Bayesian network
classifiers}\label{bayesian-network-classifiers}

\label{bayesian-networks} A Bayesian network classifier is a Bayesian
network used for predicting a discrete class variable \(C\). It assigns
\(\mathbf{x}\), an observation of \(n\) predictor variables (features)
\(\mathbf{X} = (X_1,\ldots,X_n\)), to the most probable class:

\[ c^* = \argmax_c P(c \mid \mathbf{x}) = \argmax_c P(\mathbf{x}, c).\]

\noindent The classifier factorizes \(P(\mathbf{x}, c)\) according to a
Bayesian network
\(\mathcal{B} = \langle \mathcal{G}, \boldsymbol{ \theta } \rangle\).
\(\mathcal{G}\) is a directed acyclic graph with a node per each of the
variables \((\mathbf{X}, C)\), encoding conditional independences: a
variable \(X\) is independent of its nondescendants in \(\mathcal{G}\)
given the values \(\mathbf{pa}(x)\) of its parents. \(\mathcal{G}\) thus
factorizes the joint into local conditional distributions over subsets
of variables:

\[P(\mathbf{x}, c) = P(c \mid \mathbf{pa}(c)) \prod_{i=1}^{n} P(x_i \mid \mathbf{pa}(x_i)).\]

\noindent The parameters \(\boldsymbol{ \theta }\) specify the local
distributions.

\subsection{Learning structure}\label{learning-structure}

\label{sec:bkg:learning} We learn \(\mathcal{B}\) from a data set
\(\mathcal{D} = \{ (\mathbf{x}^{1}, c^{1}), \ldots, (\mathbf{x}^{N}, c^{N}) \}\)
of \(N\) observations of \(\mathbf{X}\) and \(C\). There are two main
approaches to learning the structure \gstuc/: a) testing for conditional
independence among triplets of variables and b) searching a space of
possible structures in order to optimize a network quality score. Under
assumptions such as a limited number of parents per variable, approach
a) can produce the correct network in polynomial time
\citep{cheng-greiner02,Tsamardinos2003a}. On the other hand, finding the
optimal structure even with at most two parents per variable is NP-hard
\citep{Chickering2004}. Thus, heuristic search algorithms, such as
greedy hill-climbing, are commonly used
\citep[see, e.g.,][]{Koller2009}. Ways to reduce model complexity, in
order to avoid overfitting the training data \(\mathcal{D}\), include
searching in restricted structure spaces and penalizing it with adequate
scores.

Common scores in structure learning are the penalized log-likelihood
scores, such as the Akaike information criterion (AIC) \citep{Akaike74}
and Bayesian information criterion (BIC) \citep{Schwarz1978}. They
measure the model's fitting of the empirical distribution \pcxemp/
adding a penalty term that is a function of structure complexity. They
are decomposable with respect to \(\mathcal{G}\), allowing for efficient
search algorithms. Yet, with limited \(N\) and a large \(n\),
discriminative scores based on \pcgx/, such as conditional
log-likelihood and classification accuracy, are more suitable to the
classification task \citep{Friedman1997}. These, however, are not
decomposable according to \(\mathcal{G}\). While one can add a
complexity penalty to them \citep[e.g.,][]{grossman2004}, they are
instead often cross-validated to induce preference towards structures
that generalize better, making their computation even more time
demanding.

For Bayesian network classifiers, a common \citep[see][]{Bielza14}
structure space is that of augmented naive Bayes \citep{Friedman1997}
models (see Figure \ref{fig:structures}), factorizing
\(P(\mathbf{X}, C)\) as

\begin{equation}
P(\mathbf{X}, C) = P(C) \prod_{i=1}^{n} P(X_i \mid \mathbf{Pa}(X_i)), \label{eq:augnb}
\end{equation}

\noindent with \(C \in \mathbf{Pa}(X_i)\) for all \(X_i\) and
\(\mathbf{Pa}(C) = \emptyset\). Models of different complexity arise by
extending or shrinking the parent sets \(\mathbf{Pa}(X_i)\), ranging
from the NB \citep{Minsky1961} with \(\mathbf{Pa}(X_i) = \{C \}\) for
all \(X_i\), to those with a limited-size \(\mathbf{Pa}(X_i)\)
\citep{Friedman1997,Sahami1996}, to those with unbounded
\(\mathbf{Pa}(X_i)\) \citep{Pernkopf2003}. \textbf{While the NB can only
represent linearly} separable classes (see Minsky) \citep{jaeger2003},
more complex models are more expressive \citep{Varando2015jmlr}. Simpler
models, with sparser \(\mathbf{Pa}(X_i)\), may perform better with less
training data, due to their lower variance, yet worse with more data as
the bias due to wrong independence assumptions will tend to dominate the
error.

The algorithms that produce the above structures are generally instances
of greedy hill-climbing \citep{Keogh2002,Sahami1996}, with arc inclusion
and removal as their search operators. Some \citep[e.g.,][]{Pazzani1996}
add node inclusion or removal, thus embedding feature selection
\citep{Guyon2003} within structure learning. Alternatives include the
adaptation \citep{Friedman1997} of the Chow-Liu \citep{Chow1968}
algorithm to find the optimal one-dependence estimator (ODE) with
respect to decomposable penalized log-likelihood scores in time
quadratic in \(n\). Some structures, such as NB or AODE, are fixed and
thus require no search.

\subsection{Learning parameters}\label{learning-parameters}

Given \(\mathcal{G}\), learning \(\boldsymbol{\theta}\) in order to best
approximate the underlying \PCX/ is straightforward. For discrete
variables \(X_i\) and \(\mathbf{Pa}(X_i)\), Bayesian estimation can be
obtained in closed form by assuming a Dirichlet prior over
\(\boldsymbol{\theta}\). With all Dirichlet hyper-parameters equal to
\(\alpha\),

\begin{equation}
\theta_{ijk} = \frac{N_{ijk} + \alpha}{N_{ \cdot j \cdot } + r_i \alpha},
\label{eq:disparams}
\end{equation}

where \(N_{ijk}\) is the number of instances in \(\mathcal{D}\) such
that \(X_i = k\) and \(\mathbf{pa}(x_i) = j\), corresponding to the
\(j\)-th possible instantiation of \(\mathbf{pa}(x_i)\), while \(r_i\)
is the cardinality of \(X_i\). \(\alpha = 0\) in \req{disparams} yields
the maximum likelihood estimate of \(\theta_{ijk}\). With incomplete
data, the parameters of local distributions are no longer independent
and we cannot separately maximize the likelihood for each \(X_i\) as in
\req{disparams}. Optimizing the likelihood requires a time-consuming
algorithm like expectation maximization \citep{Dempster1977} which does
not guarantee convergence to the global optimum.

While the NB can separate any two linearly separable classes given the
appropriate \mthetas/, learning by approximating \PCX/ cannot render the
optimal \mthetas/ in some cases \citep{jaeger2003}. Multiple methods
\citep{Hall2007,Zaidi2013,Zaidi2017} learn a weight \(w_i \in [0,1]\)
for each feature and then update \(\boldsymbol{\theta}\) as

\begin{equation*}
  \theta_{ijk}^{weighted} = \frac{(\theta_{ijk})^{w_i}}{\sum_{k=1}^{r_i} (\theta_{ijk})^{w_i}}.
\end{equation*}

\noindent A \(w_i < 1\) reduces the effect of \(X_i\) on the class
posterior, with \(w_i = 0\) omitting \(X_i\) from the model, making
weighting more general than feature selection. The weights can be found
by maximizing a discriminative score \citep{Zaidi2013} or computing the
usefulness of a feature in a decision tree \citep{Hall2007}. Mainly
applied to naive Bayes models, a generalization for augmented naive
Bayes classifiers has been recently developed \citep{Zaidi2017}.

Another parameter estimation method for the naive Bayes is by means of
Bayesian model averaging over the \(2^n\) possible naive Bayes
structures with up to \(n\) features \citep{Dash2002}. It is computed in
time linear in \(n\) and provides the posterior probability of an arc
from \(C\) to \(X_i\).

\subsection{Inference}\label{inference}

Computing \pcgx/ for a fully observed \x/ means multiplying the
corresponding \(\boldsymbol{\theta}\). With an incomplete \x/, however,
exact inference requires summing over parameters of the local
distributions and is NP-hard in the general case \citep{cooper1990}, yet
can be tracktable with limited-complexity structures. The AODE ensemble
computes \pcgx/ as the average of the \(P_i (c\mid\mathbf{x})\) of the
\(n\) base models. A special case is the lazy elimination
\citep{zheng2006efficient} heuristic which omits \(x_i\) from
\req{augnb} if \(P(x_i \mid x_j) = 1\) for some \(x_j\).
